\documentclass[11pt]{article}

\usepackage{a4wide}
\usepackage[utf8]{inputenc}
\usepackage[russian]{babel}
\usepackage{graphicx}
\usepackage{amsmath}
\usepackage{float}
%\usepackage{hyperref}
\usepackage{amssymb}
\usepackage{bbm}
\usepackage{amsthm}
\usepackage{indentfirst}
\usepackage{float}
\usepackage[hidelinks]{hyperref}

%begin
\DeclareMathOperator{\sinc}{sinc}
\DeclareMathOperator{\arctg}{arctg}
\DeclareMathOperator{\sgn}{sgn}
\renewcommand{\le}{\leqslant}
\renewcommand{\ge}{\geqslant}
\newcommand{\scalar}[2]{\left<#1,#2\right>}
\newcommand{\norm}[1]{\left\lVert #1 \right\rVert}
\newcommand{\abs}[1]{\left\lvert #1 \right\rvert}

\newtheorem{theorem}{Теорема}
\newtheorem{statement}{Утверждение}
\theoremstyle{definition}
\newtheorem{definition}{Определение}[section]
%end


\begin{document}

\thispagestyle{empty}

\begin{center}
\ \vspace{-3cm}

\includegraphics[width=0.5\textwidth]{msu.eps}\\
{\scshape Московский государственный университет имени М.~В.~Ломоносова}\\
Факультет вычислительной математики и кибернетики\\
Кафедра системного анализа

\vfill

{\LARGE Отчет по курсовой работе}

\vspace{1cm}

{\Huge\bfseries <<Задание по исследованию дискретных и непрерывных
динамических систем на плоскости>>}
\end{center}

\vspace{1cm}

\begin{flushright}
  \large
  \textit{Студент 315 группы}\\
  А.\,Д.~Тулегенов

  \vspace{5mm}

  \textit{Руководитель}\\
  В.\,В.~Абрамова
\end{flushright}

\vfill

\begin{center}
Москва, 2025
\end{center}

\newpage

\section{Постановка задачи}
Во всех заданиях значения переменных \(u_t\) , \(u_{t\pm1}\) неотрицательны, значения параметров положительны. В ходе исследования системы без запаздывания необходимо:
\begin{itemize}
\item найти неподвижные точки и исследовать их устойчивость;
\item проверить существования циклов длины 2 и 3;
\item построить бифуркационную диаграмму (в смысле предельного поведения траекторий) в зависимости от значения бифуркационного параметра (значения других параметров и значение \(u_0\) следует зафиксировать при построении);
\item построить зависимость показателя Ляпунова от значения параметра.
\end{itemize}
Для системы с запаздыванием: найти особые точки и исследовать их устойчивость, проверить существование бифуркации Неймарка-Сакера, если она присутствует – построить инвариантную кривую. Исследование необходимо проводить аналитически, подкрепляя результат иллюстрациями, полученными с помощью численного моделирования.

Система без запаздывания:

\begin{equation}
    u_{t+1} = {ru^2_t}(1-\ln(1+u_t)).\label{eq:1}
\end{equation}

Система с запаздыванием:

\begin{equation}
    u_{t+1} = {ru^2_t}(1-\ln(1+u_{t-1})).\label{eq:2}
\end{equation}

\section{Система без запаздывания}
\subsection{Неподвижные точки}
\textbf{Определение.} \textit{Точка $u$ называется неподвижной точкой динамической системы $u_{t+1} = f(u)$, если $f(u) = u$.}\\
Найдём неподвижные точки (1).~Для нахождения воспользуемся равенством:
\[
u = ru^2 (1 - \ln(1 + u)).
\]
При любом значении параметра r система всегда имеет тривиальную неподвижную точку $u^{*}_0 = 0$. Для анализа других решений введем вспомогательные функции:
\[
f(u) = ru^2 (1 - \ln(1 + u)), \qquad g(u) = u.
\]
Исследуем производные.\\
\textbf{Первая производная} $f(u)$:
\begin{equation}
f'(u) = u r \left( \frac{u + 2}{u + 1} - 2 \ln(1 + u) \right).
\end{equation}
При малых $u > 0$ производная положительна, а также существует точка $u_0$, после которой производная становится отрицательной.

\textbf{Вторая производная} $f(u)$:
\begin{equation}
f''(u) = -r \left( \frac{u^2 - 2}{(u + 1)^2} + 2 \ln(1 + u) \right).
\end{equation}
Вторая производная меняет знак, что указывает на наличие точки перегиба.
Приходим к тому, что функция $f(u)$ сначала возрастает, достигает максимума, затем убывает, при $u \to +\infty$ функция стремится к $-\infty$.\\
Можно сделать вывод, что cистема имеет как минимум одну нетривиальную неподвижную точку при определённых значениях $r$, что требует дальнейшего численного или аналитического исследования.
\\\\
Выделим три качественно различных случая в зависимости от параметра r:
\begin{enumerate}
\item 1 точка пересечения \( u_0^* = 0 \);

\item 2 точки пересечения \( u_0^* = 0 \) и точка касания графиков \( f(u) \) и \( g(u) \);

\item 3 точки пересечения \( u_0^* = 0 \) и 2 точки пересечения графиков \( f(u) \) и \( g(u) \).
\end{enumerate}

При помощи программы на языке Matlab численно найдем значение параметра \( r \), при котором графики функций касаются. Для этого нужно решить систему:

\[
\begin{cases}
f(u) = g(u) \\
f'(u) = g'(u)
\end{cases}
\]

Получаем:

\[
\begin{cases}
u = 0.76322 \\
r = 3.0269
\end{cases}
\]

Следовательно, в зависимости от параметра \( r \):

\begin{enumerate}
\item 1 особая точка \( u_0^* = 0 \) при \( r < 3.0269 \);

\item 2 особых точки \( u_0^* = 0 \) и \( u_1^* = 0.76322 \) при \( r = 3.0269 \);

\item 3 особых точки \( u_0^* = 0 \) и \( 0 < u_2^* < 0.76322 < u_3^* \) при \( r > 3.0269 \) (См. рис.~\ref{fig:fixed_points}).

\newline
\end{enumerate}
\begin{figure}[H]
\includegraphics[width=1\textwidth]{1.png}
\caption{Особые точки}
\label{fig:fixed_points}
\end{figure}

\subsection{Устойчивость неподвижных точек}
\textbf{Определение.} Неподвижная точка $u^*$ \textbf{устойчива по Ляпунову}, если для любого $\epsilon > 0$ существует $\delta > 0$, такая что для любой точки $u_0$ из $\delta$-окрестности точки $u^*$ верно $|u^* - u_t| < \epsilon$ для любого $t \geq 0$, то есть траектория системы $u_t$ содержится в $\epsilon$-окрестности точки $u^*$.

\textbf{Определение.} Неподвижная точка $u^*$ \textbf{асимптотически устойчива}, если она устойчива по Ляпунову, и, кроме того, для тех же $u_0$, что и в прошлом определении, выполнено
\[
\lim_{t \to \infty} u_t = u^*.
\]

Исследуем особые точки на устойчивость. Для этого воспользуемся следующим утверждением:

\textbf{Утверждение}
Пусть \( u^* \) — особая точка системы \( u_{t+1} = f(u_t) \). Тогда

\[
\left| f'(u^*) \right| < 1 \Rightarrow u^* - \text{асимптотически устойчивая особая точка.}
\]

\[
\left| f'(u^*) \right| > 1 \Rightarrow u^* - \text{неустойчивая особая точка.}
\]
\[
\left| f'(u^*) \right| = 1 \Rightarrow\text{об устойчивости $u^*$ ничего сказать нельзя.} 
\]
В точке $u^{*}_0$ получим,что $f'(u^{*}_0) = 0$ при любом значении параметра r. То есть dточка является всегда устойчивой.\\
В точке $u^{*}_1 = 0.76322$ при $r = 3.0269$ получим, что $f'(u^{*}_1) = 1.$. То есть об устойчивости ничего сказать нельзя. Точка является странным аттрактором.\\
Для точек $0 < u_2^* < 0{,}76322 < u_3^*$ при $r > 3{,}0269$ воспользуемся программой на языке \texttt{Matlab} и посмотрим на график производной. На рис.~\ref{fig:diffs} Красным отмечены значения производных для $u^*_3, а синим для $u^*_2$. 
Производная $f'(u_2^*) > 1$ при любом $r > 3{,}0269$ — то есть точка неустойчива всегда. Производная $f'(u_3^*) < 1$ при $3{,}0269 < r < 4{,}2822$ — то есть устойчива.При $r = 4{,}2822$, $f'(u_3^*) = 1$ — то есть граница устойчивости.При $r > 4{,}2822$, $f'(u_3^*) > 1$ — то есть точка теряет устойчивость.

\begin{figure}[H]
\includegraphics[width=1\textwidth]{2.png}
\caption{Зависимость значений производной в особых точках от параметра r}
\label{fig:diffs}
\end{figure}

Отметим на графиках случаи:
На рис.~\ref{fig:case1},~\ref{fig:case2},~\ref{fig:case3} звездой указано начальное приближение, наглядно видим как меняется устойчивость в зависимости от значений $r$. 
\begin{figure}[H]
\includegraphics[width=1\textwidth]{3.png}
\caption{u* = 0 - устойчива при любом значении r}
\label{fig:case1}
\end{figure}

\begin{figure}[H]
\includegraphics[width=1\textwidth]{4.png}
\caption{r = 4 < 4,2822 - $u^*_2$ - не устойчива, $u^*_3$ - устойчива}
\label{fig:case2}
\end{figure}

\begin{figure}[H]
\includegraphics[width=1\textwidth]{5.png}
\caption{r = 4,5 > 4,2822 - $u^*_2$ и $u^*_3$ - не устойчивы}
\label{fig:case3}
\end{figure}


\subsection{Цикл длины 2 и 3}
\textbf{Определение.} \textit{Цикл длины $k$ — упорядоченный набор точек $(N_1, N_2, \ldots, N_k)$ такой, что:
\[
f(N_1) = N_2, \quad f(N_2) = N_3, \quad \ldots, \quad f(N_k) = N_1.
\]}
\subsubsection{Цикл длины 3}
Начнем проверку наличия цикла 3, потому что из его существования по теореме Шарковского будет следовать существование и цикла длины 2. Рассмотрим систему:
\[
\begin{cases}
f(f(f(u))) = u, \\
\dfrac{d}{du} f(f(f(u))) = 1.
\end{cases}
\]
для решения системы воспользуемся программой на языке \texttt{Matlab}. Получаем:
\[
r = 5{,}05, \quad 
\begin{cases}
u = 0{,}6745 = u_1, \\
u = 1{,}1131 = u_2, \\
u = 1{,}5760 = u_3.
\end{cases}
\]
На рис.~\ref{fig:fff} отмечен график функции $f(f(f(u)))$, красным отмечены точки, образующие цикл:
\begin{figure}[H]
\includegraphics[width=1\textwidth]{6.png}
\caption{при r = 5.05}
\label{fig:fff}
\end{figure}
Значит цикл длина 3 существует, и следовательно существует и цикл длины 2. Результаты согласовываются с бифуркационной диаграммой.
На рис.~\ref{fig:cycle} циклы длины 2 и 3:
\begin{figure}[H]
\includegraphics[width=1\textwidth]{7.png}
\caption{цикл длины 3, при r = 5.05}
\label{fig:cycle1}
\end{figure}
\begin{figure}[H]
\includegraphics[width=1\textwidth]{8.png}
\caption{цикл длины 2, при r = 4.4}
\label{fig:cycle2}
\end{figure}


\subsection{Бифуркационная диаграмма}
\textbf{Определение.}Появление топологически неэквивалентных фазовых портретов
 при изменении параметров динамической системы называется бифуркацией.
\textbf{Определение.} Бифуркационной диаграммой динамической системы называется раз
биение пространства параметров на максимальные связные подмножества, которые
 определяются соотношениями топологической эквивалентности и рассматриваются
 вместе с фазовыми портретами для каждого элемента разбиения.\\


Построим бифуркационную диаграмму для системы с начальной точкой $u^0 = 0.5$. Для этого будем перебирать значения параметра $r$ от 0 до 6 c шагом 0.006(см. рис.~\ref{fig:bifur}). Можно заметить, что подтверждается вывод о наличие цикла длины 3
 (и любой другой длины).

\begin{figure}[H]
\includegraphics[width=1\textwidth]{9.png}
\caption{u_0 = 0.5}
\label{fig:bifur}
\end{figure}

\subsection{Показатель Ляпунова}
%эксперимент
Показатель Ляпунова для системы определяется следующим образом:
\[
p(u_0)=\lim_{m\to\infty}\frac{1}{m}\sum_{t=1}^{m}\ln\!\bigl|f'(u_t)\bigr| ,
\]
где $u_t=f(u_{t-1})$, $t=1,2,\ldots,m$. При $p(u_0)<0$ траектории, выпущенные
из окрестности $u_0$, остаются близки к траектории, выпущенной из этой точки
$(u_0,u_1,u_2,\ldots)$; а при $p(u_0)>0$ — отдаляются от неё.

Построим график показателя Ляпунова для системы с начальной точкой
$u_0=0{,}5$ в зависимости от параметра $r$ (рис.~\ref{fig:lyapunov}).

\begin{figure}[H]
\includegraphics[width=1\textwidth]{10.png}
\caption{u_0 = 0.5}
\label{fig:lyapunov}
\end{figure}

\section{Исследование системы с запаздыванием}
Для начала преобразуем систему (2) в двумерную систему без запаздывания:
\[
    u_{t+1} = {ru^2_t}(1-\ln(1+u_{t-1})).\label{eq:2}
\]
Для начала преобразуем систему \boxed{2} в двумерную систему без запаздывания:

\[
u_{t+1} = {ru^2_t}(1-\ln(1+u_{t-1})) \Leftrightarrow 
\begin{cases}
u_{t+1} = f(u_t, v_t), \\
v_{t+1} = g(u_t, v_t),
\end{cases}
\]

где \( f(u, v) = {ru^2_t}(1-\ln(1+u_{t-1})), ~\( g(u, v) = u \).

\subsection{Особые точки системы}

Для поиска особых точек воспользуемся системой

\[
\begin{cases}
u = {ru^2}(1-\ln(1+v)), \\
v = u.
\end{cases}
\]

Получим

\[
u = {ru^2}(1-\ln(1+u))
\]
\newpage
Решения этой системы найдены в предыдущем разделе. Таким образом, получим следующие особые точки:
\begin{enumerate}
    \item 1 особая точка $(u_0^*, u_0^*)$, $u_0^* = 0$ при $r < 3.0269$;
    
    \item 2 особых точки $(u_0^*, u_0^*)$ и $(u_1^*, u_1^*)$, $u_0^* = 0$ и $u_1^* \approx 0.76322$ при $r = 3.0269$;
    
    \item 3 особых точки $(u_0^*, u_0^*)$, $(u_1^*, u_1^*)$ и $(u_2^*, u_2^*)$, $u_0^* = 0$ и $0 < u_2^* < 0.76322 < u_3^*$ при $r > 3.0269$.
\end{enumerate}

\subsection{Устойчивость неподвижных точек}
Для исследования устойчивости нам понадобится утверждение аналогичное одномерному случаю.
\textbf{Определение}
Пусть дана система \( u_{t+1}^i = f^i(u_t^1, \ldots, u_t^n), i = \overline{1,n} \). 
\[ J(u^*) = \left[ \frac{\partial f^i}{\partial u^j}(u^*)\right], \quad i, j = \overline{1,n} \]
— матрица Якоби системы в особой точке \( u^* \). Собственные значения \( J(u^*) \) называются мультипликаторами и обозначаются \( \mu_1, \ldots, \mu_n \).

\textbf{Утверждение}
\textit{Пусть \( u^* \) — особая точка системы. Тогда}
\[ |\mu_i| < 1 \quad \forall i = \overline{1,n} \Rightarrow u^* \text{ — асимптотически устойчивая особая точка.} \]
\[ \exists i \in \overline{1,n} : |\mu_i| > 1 \Rightarrow u^* \text{ — неустойчивая особая точка.} \]

Для данной системы матрица Якоби будет выглядеть следующим образом:
\[ J(u, v) = \begin{bmatrix}
2ru(1-\ln(1+v)) & -\frac{ru^2}{1+v} \\
1 & 0
\end{bmatrix}. \]
\subsubsection{при $\forall r$ для ;$(u_0^*,u_0^*)$:}
\[J(u_0^*,u_0^*)=
\begin{pmatrix}
0 & 0 \\[6pt]
1 & 0
\end{pmatrix}.\]
характеристическое уравнение $\lambda^2 = 0$ имеет корни $\lambda_1 = \lambda_2 = 0$. Следовательно, нулевая неподвижная точка является асимптотически устойчивой при любом $r$.
\subsubsection{при $r = 3.0269$ для $(u_1^*, u_1^*)$:}
Подставляя это соотношение в матрицу Якоби, получаем
\[
J(u^*,v^*) =
\begin{pmatrix}
2 & 1 \\[6pt]
1 & 0
\end{pmatrix},
\]
Получаем корни $\lambda_1 = \lambda_2 = 1$. Значит, об устойчивости в этом случае сказать ничего нельзя
\subsubsection{при $r > 3.0269$ для $(u_1^*, u_1^*)$ и $(u_2^*,u_2^*)$}

Воспользуемся программой на языке Matlab.
На рисунке~\ref{fig:eigs} приведены зависимости собственных значений $\lambda_1(r)$ и $\lambda_2(r)$ от параметра $r$ для обеих ветвей ненулевых неподвижных точек $u_1^*(r)$ и $u_2^*(r)$. Из графика видно, что для всех $r$, где существуют ненулевые особые точки, хотя бы одно из собственных значений по модулю не меньше единицы. В точке касания $r=r_{\min}$ оба собственных значения равны $1$, а при $r>r_{\min}$ одно из них превышает $1$ по модулю. 

Таким образом, \textbf{все ненулевые неподвижные точки являются неустойчивыми}, независимо от значения параметра $r$. Единственной асимптотически устойчивой особой точкой системы остаётся начало координат $(0,0)$.

\begin{figure}[H]
    \centering
    \includegraphics[width=0.85\textwidth]{11.png}
    \includegraphics[width=0.85\textwidth]{12.png}
    \caption{Собственные значения $\lambda_1(r), \lambda_2(r)$ ненулевых неподвижных точек $u_1^*(r), u_2^*(r)$}
    \label{fig:eigs}
\end{figure}
График наглядно показывает, что при $r = r_{\min} \approx 3.0269$
ненулевые неподвижные точки находятся на границе устойчивости
($|\lambda|=1$), а при $r > r_{\min}$ хотя бы одно из собственных значений
превышает $1$ по модулю, что означает неустойчивость этих точек.

\subsection{Бифуркация Неймарка-Сакера}
\textbf{Определение.} Бифуркация Неймарка-Сакера --- бифуркация положения равновесия в системе, соответствующая паре собственных значений $|\lambda_1| = |\lambda_2| = 1$, $\lambda_1 = \overline{\lambda_2}$:

\begin{figure}[H]
\centering
\includegraphics[width=1\textwidth]{13.png}
\caption{Появление двух ненулевых особых точек при $r = 3.0269}
\label{fig:new_points}
\end{figure}

На рис.~\ref{fig:new_points} показано появление двух ненулевых особых точек
при $r = r_{\min} \approx 3.0269$. При этом количество решений уравнения
возрастает с одной (нулевая точка) до трёх. Данный график подтверждает, что
при $r = r_{\min}$ действительно возникают новые неподвижные точки.
\newline
Для проверки наличия бифуркации Неймарка–Сакера необходимо рассматривать
собственные значения матрицы Якоби. Так как они действительные и не образуют
комплексную сопряжённую пару, пересекающую единичную окружность, бифуркация
Неймарка–Сакера отсутствует.

\section*{Литература}
\item Абрамова В.\,В. Лекции по динамическим системам и биоматематике. 2024.
\item Братусь А.\,С., Новожилов А.\,С., Платонов А.\,П. \textit{Динамические системы и методы в биологии}. — М.: ФИЗМАТЛИТ, 2010.
\end{enumerate}


\end{document}